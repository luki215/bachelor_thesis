\chapter{Requirements}




\begin{itemize}
	\item formalize company structure
	\item formalize the process of order and design the application architecture based on it
	\item authorization
	\item We will provide the REST API using which the front-end applications will communicate. We focus on unifying individual company entities under one endpoint.
	\todo{dopsat?}
	on clear and uniform naming conventions 
	\item effective scheduler
	\item bug reporting, back-up, monitoring, deploy process
\end{itemize}





In the first part of the chapter we focus on the requirements for the wholesome application solution and related business requirements. In the second part we specify the technical point of view.

\section{Business requirements}
\todo{poznámky}
Začít s nějakou vizí, jakým způsobem jsme se k těm informacím dostal, že jsme to rozdělili do nějakých kategorií a pak popis těchhle kategorií.





\section{Technical requirements}
	Our first requirement is to have the back-end and front-end strictly separated. This allows us in the future having more different clients\todo{špatné pojmenování - client používán jako zákazník všude jinde} which can be connected to our logic. With this architecture we can easily integrate ordering system into common applications like Facebook Messenger or Slack and do not relay on custom application only.
	
	\todo{poznámky}
	To nahoře není moc dobrý requirement - nco co se dá testovat, např. musí to být použitelné z mobilu i z webu.
	
	well documented vyhodit - nesmysl, psát jen požadavky, jakože bude více klientů, kteří s tím budou komunikovat. To samé bezpečnost - https není třeba zmiňovat.
	
	internal requirements - frontend dát kdyžtak až nakonec. Nedávat i to s OS - dát třeba do requirements na development process
	
	Napsat hlavně EXTERNÍ requirementy
	
	Our API must be well-documented and easy to use so the front-end part can relay on the information in documentation. The communication with API must be encrypted - we should use the HTTPS protocol. 
	
	Because we have other developers working on the front-end side, we should keep the installation process of the back-end as simple as possible. We should be able to run it on various operating systems - specifically latest version of Windows, Ubuntu and MacOS X.
	
	The stack we use for back-end application should be also prepared for scaling in the future and keep the downtime during the deployment at minimum.