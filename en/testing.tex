\chapter{Testing}
In the following chapter we describe process of testing the application. Each feature and request has been tested by us after the implementation. Then we published the feature to front-end. Naturally during the implementation they tested the feature again.

 In our application we evaluated two critical parts to which we decided adding third layer of testing. First is the authentication and user manipulation part, second is the order scheduler. For these features we followed the \textit{Test-driven development}\footnote{\url{https://en.wikipedia.org/wiki/Test-driven_development}} software development process. When all of the tests for given feature have passed we continued testing the feature such as the other parts of the application.

 For the user part of the application we used integration tests. From our experience the process of creating and user activation is almost never revisited during the front-end development thus we loose the the front-end insurance part. Furthermore the user registration and activation process is the most crucial part for the business. When there occurs error during the registration, it is likely that the user will never use our application again.  
  
 The reason why we decided to cover the scheduler system was different. First of all we started by analysis of the possible situations. During the development we discovered more and more edge-cases and by fixing them we started to breaking other parts. Also reproduce all the edge cases took a lot of time. Also for the route calculation we use the Google Maps API, which even for the same two points gives different time estimations based on current traffic. In the end we decided to mock the API and write set of tests covering all of the possible situations and desired outputs. Based on these tests we started to implement the scheduler. 
 

\section{Technologies used for testing}

Ruby on Rails has rich support libraries which make the testing easy. As the testing library core we decided to use the \textit{RSpec}\footnote{\url{http://rspec.info/}}. This library has intuitive interface, helpful features and is kind of standard in the Ruby on Rails world.

Next essential part of our testing stack is  the \textit{factory\_bot} gem\footnote{\url{https://github.com/thoughtbot/factory_bot}}. In a short way this library allows us to define multiple factories for each model in the application. When we need an instance of some model, we just use the library's factory and it will return the model instance with the data randomized as described in the factory definition. Also it allows us to specify relations between the factories so that we can easily get an instance of the user with three orders.  For the random data generation we use the gem \textit{faker}, which helps us with generating realistic looking random data.